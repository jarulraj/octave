\documentclass{article}

\usepackage{hw}
\usepackage{bm}
\usepackage{amsmath}
\usepackage{graphicx}
\usepackage[colorlinks=true,urlcolor=blue]{hyperref}
\usepackage{geometry}
\geometry{margin=1in}
\usepackage{multicol}
\usepackage{paralist}
\usepackage{todonotes}
\setlength{\marginparwidth}{2.15cm}
\usepackage{booktabs}
\usepackage{enumitem}
\usepackage{amsmath}
\usepackage{bm}
\usepackage{cleveref}
\usepackage{amsmath}

%\DeclareMathOperator*{\argmin}{arg\,min}
%\DeclareMathOperator*{\argmax}{arg\,max}

\def \issoln {1}

% Some commands to allow solutions to be embedded in the assignment file.
\ifcsname issoln\endcsname \else \def\issoln{0} \fi
\newcommand{\soln}[1]
{
  \if\issoln 1
  \textbf{Solution:}
  #1
  \fi
}

\begin{document}

\section*{}
\begin{center}
  \centerline{\textsc{\LARGE Homework 3{\if\issoln 1 Solutions \else \fi}}}
  \vspace{0.5em}
  \centerline{\textsc{\Large Regression, Gaussian Processes, And Boosting}}
  \vspace{1em}
  \textsc{\large CMU 10-701: Machine Learning (Fall 2014)} \\
  \url{https://piazza.com/cmu/fall2014/1070115781/home}
  \vspace{3em}
  \centerline{\large{Joy Arulraj (jarulraj)}}
  \vspace{1em}
\end{center}

\if\issoln 1 \else
\section*{START HERE: Instructions}

\begin{itemize*}


\item \textbf{Late days:} The homework is due Thursday September 25th, at 11:59PM. You have five late days to use throughout the semester, and may use at most three late days on any one assignment. Once the allowed late days are used up, each additional day (or part of a day) will subtract 1 from the normalized score for the assignment. View the full late days policy on \href{https://piazza.com/class/hxwaa1bxuze4xj?cid=10}{Piazza}.

\item \textbf{Collaboration policy:} Collaboration on solving the homework is allowed, after you have thought about the problems on your own.  It is also OK to get inspiration (but not solutions) from books or online resources, again after you have thought about the problems on your own.  There are two requirements: first, cite your collaborators and resources fully and completely (e.g., ``Jane explained to me what is asked in Question 3.4'' or ``I found an explanation of conditional independence on page 17 of Mitchell's textbook'').  Second, write up your solution independently: close the book and all of your notes, and send collaborators out of the room, so that the solution comes directly from you and you alone.

\item \textbf{Programming:} 
\begin{itemize*}
\item \textbf{Octave:} You must write your code in Octave. Octave is a free scientific programming language, with syntax identical to that of MATLAB. Installation instructions can be found on the \href{http://www.gnu.org/software/octave/}{Octave website}. (You can develop your code in MATLAB if you prefer, but you \emph{must} test it in Octave before submitting, or it may fail in the autograder.)
\item \textbf{Autograding:} All programming problems are autograded using the CMU Autolab system. The code which you write will be executed remotely against a suite of tests, and the results used to automatically assign you a grade. To make sure your code executes correctly on our servers, you should avoid using libraries which are not present in the \emph{basic} Octave install.
\end{itemize*}

\item \textbf{Submitting your work:} All answers will be submitted electronically through the submission website: \url{https://autolab.cs.cmu.edu/10701-f14}. 
\begin{itemize*}
\item Start by downloading the \href{https://autolab.cs.cmu.edu/10701-f14/attachments/view/229}{submission template}. The template consists of directory with placeholders for your writeup (``problem1.pdf'', ``problem2.pdf''), and a single sub-directory for the programming parts of the assignment. \emph{Do not modify the structure of these directories or rename these files.} 
\item \textbf{IMPORTANT:} When you download the template, you should confirm that the autograder is functioning correctly by compressing and submitting the directory provided. This should result in a grade of zero for all programming questions, and an unassigned grade (-) for the written questions. 
\item \textbf{Writeup:} Replace the placeholders with your actual writeup. Make sure to keep the expected file names (``problem1.pdf''), and to submit one PDF per problem.  To make PDFs, we suggest pdflatex, but just about anything (including handwritten answers) can be converted to PDF using copier-scanners like the ones in the copier rooms of GHC.
\item \textbf{Code:} For each programming sub-question you will be given a single function signature. You will be asked to write a single Octave function which satisfies the signature. In the handout linked above, the ``code'' folder contains stubs for each of the functions you need to complete. 
\item \textbf{Putting it all together:} Once you have provided your writeup and completed each of the function stubs, compress the top level directory \emph{as a tar file} and submit to Autolab online (URL above). You may submit your answers as many times as you like. You will receive instant feedback on your autograded problems, and your writeups will be graded by the instructors once the submission deadline has passed. 
\end{itemize*}

\end{itemize*}

\fi

%\section*{Problem 1: Gaussian processes [Abu - 40pts]}

\begin{enumerate}

	\item{
	\textbf{Regression model:}\\
	
	\vspace{0.2in}		
	f(X) $\sim$ N(0, k(X,X) and Y $\sim$ N(f(X), $\sigma^2$I)

	The following covariance functions are considered:
	\begin{itemize}
	\item{Linear : $k(x,x') = x.x'$}
	\item{Squared Exponential : $k(x,x') = exp[\frac{1}{2\lambda^2}(x-x')^2$] }
	\item{Rational quadratic : $k(x,x') = (1 + (x-x')^2)^-\alpha$}	
	\end{itemize}

	\begin{figure}[h!]
	\centering
	\includegraphics[width = 0.45\textwidth]{images/3-a-Linear.pdf}
	\hfill
	\includegraphics[width = 0.45\textwidth]{images/3-a-Squared-Exponential.pdf}
	\hfill
	\includegraphics[width = 0.45\textwidth]{images/3-a-Rational-Quadratic.pdf}
	\caption{1(a). Plot consisting of 3 sample functions, mean function and confidence band for (a) Linear, (b) Squared Exponential, and (c) Rational quadratic.}
	\label{fig:1a}
	\end{figure}	
	
	The plot is shown in \cref{fig:1a}. The point-wise variance is obtained from the diagonal of the kernel matrix.					
	}
	
	\item{	
	\textbf{Impact of noise parameter $\sigma$:}\\
	
	We pick the 	squared exponential function. The plot is shown in \cref{fig:1b}.  When we increase the Gaussian noise parameter $\sigma$, the output Y has higher variance. The curves are more spiky at higher $\sigma$ values. This makes sense since this is the noise parameter in the model.
	
	\begin{figure}[h!]
	\centering
	\includegraphics[width = 0.6\textwidth]{images/3-b-Squared-Exponential.pdf}
	\caption{1(a). Relationship between noise parameter $\sigma$ and output $Y$ for the squared exponential covariance function.}
	\label{fig:1b}
	\end{figure}		
	}	

	\item{ 
	\textbf{Conditional distribution:}\\
	$x \sim N(\mu , \Sigma)$ 
	
	$\begin{bmatrix}
	x_{1} \\
	x_{2}
	\end{bmatrix} \sim N ( 
	\begin{bmatrix}
	\mu_{1} \\
	\mu_{2}
	\end{bmatrix} , \begin{bmatrix}
	\Sigma_{11} & \Sigma_{12} \\
	\Sigma_{21} & \Sigma_{22}
	\end{bmatrix} )$\\
	
	We want to derive $p(x_{1} | x_{2})$. We will show that $p(x_{1} | x_{2}) \sim N ( \bar{\mu}, \bar{\Sigma})$.\\
	
	First, let us denote $\Sigma_{11}$, $\Sigma_{22}$, and $\Sigma_{21}$ by $A$, $B$ and $C$. Since, the covariance matrix is symmetric, $\Sigma_{12}$ is $B'$. Using \textbf{Schur's complement}, we can derive $\bar{\mu}$ and $\bar{\Sigma}$.\\
	
	$\begin{bmatrix}
	A & C' \\
	C & B
	\end{bmatrix} \begin{bmatrix}
	I & 0 \\
	-B^{-1}C & I
	\end{bmatrix} =
	\begin{bmatrix}
	A - C'B^{-1}C & C' \\
	0 & B
	\end{bmatrix}$\\
		
	$\begin{bmatrix}
	A - C'B^{-1}C & C' \\
	0 & B
	\end{bmatrix}
	\begin{bmatrix}
	(A - C'B^{-1}C)^{-1} & 0 \\
	0 & B^{-1}
	\end{bmatrix} =
	\begin{bmatrix}
	I & B^{-1}C' \\
	0 & I
	\end{bmatrix}$\\

	$\begin{bmatrix}
	I & B^{-1}C' \\
	0 & I
	\end{bmatrix}
	\begin{bmatrix}
	I & -C'B^{-1} \\
	0 & I
	\end{bmatrix} =
	\begin{bmatrix}
	I & 0 \\
	0 & I
	\end{bmatrix}$\\

	Therefore, \\
	
	$\begin{bmatrix}
	A & C' \\
	C & B
	\end{bmatrix}^{-1}
	=
	\begin{bmatrix}
	I & 0 \\
	-B^{-1}C & I
	\end{bmatrix}
	\begin{bmatrix}
	(A - C'B^{-1}C)^{-1} & 0 \\
	0 & I
	\end{bmatrix}
	\begin{bmatrix}
	I & -C'B^{-1} \\
	0 & I
	\end{bmatrix}$\\

	Now, the conditional distribution can be written as:\\
	
	$p(x_{1}, x_{2}) \propto exp( -0.5 * 
	\begin{bmatrix}
	x_{1} - \mu_{1} \\
	x_{2} - \mu_{2} 
	\end{bmatrix}'
	\begin{bmatrix}
	A & C' \\
	C & B   
	\end{bmatrix}^{-1} 
	\begin{bmatrix}
	x_{1} - \mu_{1} \\
	x_{2} - \mu_{2} 
	\end{bmatrix} )$ \\

	Expanding the inner matrix using schur's complement, we get:\\

$p(x_{1}, x_{2}) \propto exp( -0.5 * 
	\begin{bmatrix}
	x_{1} - \mu_{1} \\
	x_{2} - \mu_{2} 
	\end{bmatrix}'
	\begin{bmatrix}
	I & 0 \\
	-B^{-1}C & I   
	\end{bmatrix} 
	\begin{bmatrix}
	(A-C'B^{-1}C)^{-1} & 0 \\
	0 & B^{-1}   
	\end{bmatrix} 
	\begin{bmatrix}
	I & -C'B^{-1} \\
	0 & I   
	\end{bmatrix}
	\begin{bmatrix}
	x_{1} - \mu_{1} \\
	x_{2} - \mu_{2} 
	\end{bmatrix} )$ \\	
	
	Multiplying the first two and last two matrices,\\

$p(x_{1}, x_{2}) \propto exp( -0.5 * 
	\begin{bmatrix}
	x_{1} - \mu_{1} - C'B^{-1} (x_{2} - \mu_{2}) \\
	x_{2} - \mu_{2} 
	\end{bmatrix}'
	\begin{bmatrix}'
	(A-C'B^{-1}C)^{-1} & 0 \\
	0 & B^{-1}   
	\end{bmatrix} 
	\begin{bmatrix}
	x_{1} - \mu_{1} - C'B^{-1} (x_{2} - \mu_{2}) \\
	x_{2} - \mu_{2} 
	\end{bmatrix} )$\\ 	
		
	As the center matrix is block diagonal, we rewrite this as:\\		

$p(x_{1}, x_{2}) \propto exp( -0.5 * 
	(x_{1} - \mu_{1} - C'B^{-1} (x_{2} - \mu_{2}))'
	(A-C'B^{-1}C)^{-1}
	(x_{1} - \mu_{1} - C'B^{-1} (x_{2} - \mu_{2}))) . exp( -0.5 * (x_{2} - \mu_{2})'B^{-1}(x_{2} - \mu_{2}) )$\\

	Conditioning on $x_{2}$,	 we have :\\		
	
	$p(x_{1} | x_{2}) \propto exp( -0.5 * 
	(x_{1} - \mu_{1} - C'B^{-1} (x_{2} - \mu_{2}))'
	(A-C'B^{-1}C)^{-1}
	(x_{1} - \mu_{1} - C'B^{-1} (x_{2} - \mu_{2}))) $\\
		
	Therefore, $p(x_{1} | x_{2}) \sim N ( \bar{\mu}, \bar{\Sigma})$, where\\
	
	$\bar{\mu} = \mu_{1} + C'B^{-1}(x_{2} - \mu_{2}) =  \mu_{1} + \Sigma_{12}\Sigma_{22}^{-1}(x_{2} - \mu_{2}) \\
	\bar{\Sigma} = A - C'B^{-1}C = \Sigma_{11} - \Sigma_{12}\Sigma_{22}^{-1}\Sigma_{21}$\\
	
	
	This gives the required conditional probability distribution.
	}
	
	\item{
	Distribution of $f(X) | Y_{*}$ :\\
	
	$\begin{bmatrix}
	f(X) \\
	Y_{*}
	\end{bmatrix} \sim N ( 0, 
	\begin{bmatrix}
	k(X,X) & k(X,X_{*}) \\
	k(X_{*},X) & k(X_{*},X_{*}) + \sigma^{2}I	
	\end{bmatrix} )$	\\	
		
	Using the result in previous part, we substitute the following :\\
	
	$\mu_{1} = 0$, $\mu_{2} = 0$, $\Sigma_{11} = k(X,X)$, $\Sigma_{12} = k(X,X_{*})$, $\Sigma_{21} = k(X_{*},X)$, $\Sigma_{22} = k(X_{*},X_{*}) + \sigma^{2}I$, and $x_{2} = Y_{*}$\\
		
	$f(X) | Y_{*} \sim N ( \bar{\mu}, \bar{\Sigma})$,	where\\	
	
	$\bar{\mu} = \mu_{1} + C'B^{-1}(x_{2} - \mu_{2}) =  \mu_{1} + \Sigma_{12}\Sigma_{22}^{-1}(x_{2} - \mu_{2})\\
		= 0 +  k(X,X_{*})(k(X_{*},X_{*}) + \sigma^{2}I)^{-1} (Y_{*} - 0) \\
		= k(X,X_{*})(k(X_{*},X_{*}) + \sigma^{2}I)^{-1}Y_{*}$\\

	$\bar{\Sigma} = A - C'B^{-1}C = \Sigma_{11} - \Sigma_{12}\Sigma_{22}^{-1}\Sigma_{21} \\
		= k(X,X) - (k(X,X_{*})(k(X_{*},X_{*}) + \sigma^{2}I)^{-1}k(X_{*},X)$\\

	This matches the distribution shown in the problem. 			
	}
	
	\item{
		\textbf{Sampling functions from $p(f(X) | Y_{*})$:}\\
		
		The additional information we have are the training points $X_{*}$ and $Y_{*}$. We use the same covariance functions in part a. 
		\begin{itemize}
		\item{Linear : $k(x,x') = x.x'$}
		\item{Squared Exponential : $k(x,x') = exp[\frac{1}{2\lambda^2}(x-x')^2$] }
		\item{Rational quadratic : $k(x,x') = (1 + (x-x')^2)^-\alpha$}	
		\end{itemize}

		\begin{figure}[h!]
		\centering
		\includegraphics[width = 0.45\textwidth]{images/3-e-Linear.pdf}
		\hfill
		\includegraphics[width = 0.45\textwidth]{images/3-e-Squared-Exponential.pdf}
		\hfill
		\includegraphics[width = 0.45\textwidth]{images/3-e-Rational-Quadratic.pdf}
		\caption{1(a). Plot consisting of 3 sampled functions from $p(f(X) | Y_{*})$, the mean function and confidence band for (a) Linear, (b) Squared Exponential, and (c) Rational quadratic.}
		\label{fig:1e}
		\end{figure}	
		
		The plot is shown in \cref{fig:1e}. The point-wise variance is obtained from the diagonal of the kernel matrix in part d.
		We observe that the functions pass through the points in the non-linear kernel functions. This shows that they make use of the training points.			
	}	

	\item{
		\textbf{$\lambda$ parameter in squared exponential function:}\\
		
		The plot in \cref{fig:1f} shows the impact of $\lambda$ on the covariance function. Overall, we observe that as this parameter is increased, the bias increases and the variance decreases, because the confidence band is smaller but the curves don't pass through the training points at higher $\lambda$ values. This makes sense because this parameter is in the denominator of the fraction within the exponential, thereby diminishing the impact of difference between $x$ and $x'$.
		
		\begin{figure}[h!]
		\centering
		\includegraphics[width = 0.45\textwidth]{images/3-f-Squared-Exponential1.pdf}
		\hfill
		\includegraphics[width = 0.45\textwidth]{images/3-f-Squared-Exponential2.pdf}
		\hfill
		\includegraphics[width = 0.45\textwidth]{images/3-f-Squared-Exponential3.pdf}
		\caption{1(a). Plot consisting of 3 sampled functions from $p(f(X) | Y_{*})$, the mean function and confidence band for Squared Exponential kernel function with different $\lambda$ parameters.}
		\label{fig:1f}
		\end{figure}			
	
	}
	
\end{enumerate}

\section*{Problem 2: Regression [Zichao - 30 pts]}

\begin{enumerate}
	\item{
	\textbf{Why Lasso Works ?}\\
	
	The optimal parameter vector is given by :\\	

	$\beta^{*} = \argmin_{\beta}  \frac{1}{2} \| y - X\beta \|^{2} + \lambda \| \beta \|_{1}$\\
	
	The $L_{1}$ norm is used in the penalty function in lasso regression. This tends to generate sparse $\beta^{*}$ as we will show below.
	
	\begin{enumerate}	
	\item{
	We have :\\
		
		$J_{\lambda}(\beta) = \frac{1}{2} \| y - X\beta \|^{2} + \lambda \| \beta \|_{1} \\
		=  \frac{1}{2} ( Y - X\beta)'(Y - X\beta) + \lambda \| \beta \|_{1} \\
		=  \frac{1}{2} ( Y'Y - \beta'X'Y - Y'X\beta + \beta'X'X\beta)  + \lambda \| \beta \|_{1}$\\
		
	As the training data is whitened, $X'X = I$. Therefore,\\
		
		$J_{\lambda}(\beta) =  \frac{1}{2} ( Y'Y - \beta'X'Y - Y'X\beta + \beta'\beta) + \lambda \| \beta \|_{1} \\
		= \frac{1}{2} (Y'Y) + \frac{1}{2} (\beta'\beta - 	\beta'X'Y - Y'X\beta ) +  \lambda \| \beta \|_{1} $\\
		
	This can be expanded as : \\
	
		$J_{\lambda}(\beta) = \frac{1}{2} (y'y) + \sum\limits_{i=1}^d ( \frac{1}{2} * \beta_{i}^{2} - \beta_{i} * y'X_{i} + \lambda * |\beta_{i}|) \\
		= g(y) + \sum\limits_{i=1}^d f(X_{i},y,\beta_{i},\lambda) $\\
	
	Hence, this shows that $\beta_{i}^{*}$ is determined by the $i^{th}$ feature and the output and not by the other features in X.		
	}		
	
	\item{
	To find the optimal parameter vector $\beta_{i}^{*}$, let's differentiate $J_{\lambda}(\beta)$ with respect to $\beta$	 and set it to 0. Let's first consider the least-squares solution $\beta_{LS}^{*}$. We obtain :\\

	$\frac{d J_{\lambda}(\beta)}{d \beta} = -Y'X + \beta = 0 $ \\ 
	
	Thus, $\beta_{LS}^{*} = Y'X $.\\ 
	
	Now, if we add in the $L_{1}$ penalty term,\\
	
	$\frac{d J_{\lambda}(\beta)}{d \beta} = -Y'X + \beta + \lambda * \frac{d  \| \beta \|_{1}}{\beta}$ \\
	
	When  $\beta_{i}^{*} > 0$, we need $\beta_{i} > 0$ since otherwise the sign of the loss function can be changed to get lower value. Now, the derivative of $|\beta|$ with respect to $\beta$ is therefore +1. Thus,\\
	
	$\frac{d J_{\lambda}(\beta)}{d \beta} = -Y'X + \beta + \lambda = 0$\\
	
	Thus, $\beta_{i}^{*} = y'X_{i} - \lambda = \beta_{LS,i}^{*} - \lambda$.\\
	
	We require this solution to be non-negative. Therefore, we can also  rewrite this as  $\beta_{i}^{*} = sign(\beta_{LS,i}^{*})(|\beta_{LS,i}^{*}| - \lambda)^{+}$.\\
	}	
	
	\item{	
	When  $\beta_{i}^{*} < 0$, we need $\beta_{i} < 0$ since otherwise the sign of the loss function can be changed again to get lower value. Therefore, the derivative of $|\beta|$ with respect to $\beta$ is -1. Thus,\\
	
	$\frac{d J_{\lambda}(\beta)}{d \beta} = -Y'X + \beta - \lambda = 0$\\
	
	Thus, $\beta_{i}^{*} = y'X_{i} + \lambda = \beta_{LS,i}^{*} + \lambda$\\
	
	Again, we can rewrite this as  $\beta_{i}^{*} = sign(\beta_{LS,i}^{*})(|\beta_{LS,i}^{*}| - \lambda)^{+}$, because we require $\beta_{i} < 0$. \\					
	}
	
	\item{		
	When	  $\lambda > max_{i} | \beta_{i}^{LS} |$, i.e. for large values of $\lambda$, the forms derived above can hold only when $\beta_{LS,i}^{*} = 0$ for all i. Therefore, this makes $\beta_{*}$ a \textbf{sparse vector}. Note that the derivative of $|\beta|$ with respect to $\beta$ is not defined when it 0.	\\
	
	When  $\lambda = 0$, clearly $\beta_{i}^{*} = \beta_{LS,i}^{*}$ for all i. That is, it is same as the least squares solution.\\	
	}

	\item{
		\textbf{Ridge regression:}
		
		$\beta^{*} = \argmin_{\beta}  \frac{1}{2} \| y - X\beta \|^{2} + \frac{1}{2} \lambda \| \beta \|_{2}^{2}$\\
		
		Differentiating $J_{\lambda}(\beta)$ with respect to $\beta$ again, we get :\\
		
		$-Y'X + X'X\beta + \lambda * \beta = 0$\\

		As the data is whitened, \\
		
		$-Y'X + I\beta + \lambda * \beta = 0$\\
			
		Now, $\beta_{i}^{*} = \frac{y'X_{i}}{(1 + \lambda)} = \frac{\beta_{LS,i}^{*}}{1 + \lambda}$. \\
		
		Thus, when $\lambda$ tends to $\infty$, we get a \textbf{sparse} $\beta^{*}$ vector. And when it tends to 0, we get the least squares solution, i.e. $\beta_{i}^{*} = \beta_{LS,i}^{*}$ for all i.	This explains why it is difficult to get a sparse vector and do automatic variable selection in ridge regression compared to lasso regression.	\\	
	}
	
	\end{enumerate}
	}					

	\item{
	\textbf{Bayesian regression and Gaussian process}\\

	The regression model is :\\

	$f(X) = \phi'w$ and $Y = f(X) + \epsilon$\\
	
	$\epsilon \sim N(0, \sigma_{n}^{2}I)$\\
	
	$w \sim N(0, \Sigma_{p})$, where $\Sigma_{p} = \sigma_{o}^{2}I$ \\
			
	\begin{enumerate}
	\item{
		\textbf{(a) Posterior distribution :}\\
		
		$ p(w | X,Y) \propto P(Y |X,w) * p(w)$, since $p(Y |X)$ is a constant term.\\		

		Now, $w \sim N(0, \Sigma_{p})$. When X and w are fixed, $Y \sim N(f(X), \sigma_{n}^{2})$.\\
		
		$ p(w|X,Y)$ is product of gaussians.\\  
			
		$ p(w|X,Y) \propto exp(-0.5*(w- \bar{w} )'(\sigma_{n}^{-2}XX' + \Sigma_{p}^{-1})(w- \bar{w} ))$\\			

		Thus, 	$ p(w|X,Y) \sim N (\bar{\mu}, \bar{\Sigma}$, where :\\

		$\bar{\Sigma	} = (\sigma_{n}^{-2}XX' + \Sigma_{p}^{-1})^{-1}$\\	

		$\bar{\mu} = \sigma_{n}^{-2}(\sigma_{n}^{-2}XX' + \Sigma_{p}^{-1})^{-1}XY$\\		
		
		This the required mean and covariance of this distribution.
		
		\textbf{(b) Predictive distribution :}\\

		$p(f_{*} | X_{*},X,Y) = \int p(f_{*} | X_{*},w) p(w| X,Y) dw$\\
		
		Now, we already have the distribution of $p(w| X,Y)$. When $X_{*}$ and $w$ are fixed, $f_{*} = f(X_{*}) = \Phi(X_{*})'w$.
		
		Thus, this is a linear transformation of the gaussian conditioned on the given variables. Using the \textbf{affine transformation} of the posterior gaussian distribution, the mean is that of the posterior distribution multiplied by $X_{*}$ and the variance is the quadratic form of $X_{*}$ with the posterior distribution's $\bar{\Sigma}$.\\
		
		This is because if $x \sim N(\mu, \Sigma)$ and $y = c + Bx$, then $y \sim N(c + \mu*B, B\Sigma B')$. Thus, \\
		
		$p(f_{*} | X_{*},X,Y) \sim N(\bar{\mu}, \bar{\Sigma})$, where :\\
		
		$\bar{\Sigma	} = X_{*}'Z^{-1}X_{*}$\\	

		$\bar{\mu} = \sigma_{n}^{-2} X_{*}'Z^{-1}XY_{*}$\\	
		
		$ Z  = 	(\sigma_{n}^{-2}XX' + \Sigma_{p}^{-1})$
		
		This the required mean and covariance of this distribution. We note the symmetry between the predictive and  posterior distribution.
	}		

	\item{
		\textbf{Regression from Gaussian process perspective:}\\
		
		The kernel function is $k(x,x') = \sigma_{0}^{2} \phi(x)'\phi(x')$\\

		The result from the previous problem is :\\
		
		$\bar{\mu} = k(X,X_{*})(k(X_{*},X_{*}) + \sigma_{n}^{2}I)^{-1}Y_{*}$\\
		
		$\bar{\Sigma} = k(X,X) - (k(X,X_{*})(k(X_{*},X_{*}) + \sigma_{n}^{2}I)^{-1}k(X_{*},X)$\\
			
		The predictive distribution in the projected space is :\\
		
		$p (f_{*} | X_{*},X,Y) = N(\bar{\mu}, \bar{\Sigma}) = N( \sigma_{n}^{-2} \Phi(X_{*})' Z^{-1} \Phi(X)Y, \Phi(X_{*})'Z^{-1}\Phi(X_{*}) )$\\
		
		Now, as the kernel function can be written as $K = \sigma_{0}^{2}\Phi'\Phi$, we can rewrite this as :\\
		
		$p (f_{*} | X_{*},X,Y) = N(\bar{\mu}, \bar{\Sigma})$ where:\\
		
		$\bar{\mu} =  \Phi_{*}'\Sigma_{p}\Phi(K+\sigma_{n}^{2}I)^{-1}Y_{*} $\\
		
		$\bar{\Sigma} =  \Phi_{*}'\Sigma_{p}\Phi_{*} - \Phi_{*}'\Sigma_{p}\Phi(K+\sigma_{n}^{2}I)^{-1}\Phi'\Sigma_{p}\Phi_{*}$\\
		
		This is the required predictive distribution.			
	}	
	
	\item{
		\textbf{Mean of gaussian distribution $p (f_{*} | X_{*},X,Y)$ is equivalent in the two previous parts:}\\
		
		We have $\mu_{1} = \Phi_{*}'\Sigma_{p}\Phi(K+\sigma_{n}^{2}I)^{-1}Y_{*}$ and $\mu_{2} = \sigma_{n}^{-2} \Phi_{*}' Z^{-1} \Phi Y{*}$.\\
		
		By expanding the kernel,\\
		
		$\mu_{1} =  \Phi_{*}' \Sigma_{p}\Phi(K+\sigma_{n}^{2}I)^{-1}Y_{*} \\
		=  \Phi_{*}' \Sigma_{p}\Phi ( \sigma_{0}^{2}\Phi'\Phi +\sigma_{n}^{2}I)^{-1} Y_{*} \\
		=  \Phi_{*}' \Sigma_{p}\Phi ( \Phi'\Sigma_{p}\Phi +\sigma_{n}^{2}I)^{-1} Y_{*} \\
		=  \Phi_{*}' \Sigma_{p}( \Phi\Phi'\Sigma_{p}\Phi + \Phi\sigma_{n}^{2}I)^{-1} Y_{*} \\
		=  \sigma_{n}^{-2} \Phi_{*}' \Sigma_{p}(\sigma_{n}^{-2}\Phi'\Sigma_{p}\Phi + I)^{-1} \Phi Y_{*} \\
		=  \sigma_{n}^{-2} \Phi_{*}' (\sigma_{n}^{-2}\Phi'\Phi + \Sigma_{p}^{-1}I)^{-1} \Phi Y_{*} \\
		=  \sigma_{n}^{-2} \Phi_{*}' Z^{-1} \Phi Y_{*} \\
		= \mu_{2} $	
		
		Thus, the two distributions are equivalent.
	}	
	
	\item{
		\textbf{Form used in prediction : }\\
		
		When $D > n$, the number of features is large. Inverting the $D$x$D$ covariance matrix is more expensive than inverting the $n$x$n$ matrix. Thus, we use the kernelized form in part 2. When $D < n$, there are more training points. In this case, we can instead use the predictive distribution in part 1.
	
	}
	
	\end{enumerate}

	}	
\end{enumerate}

\end{document}
